%!TEX root = ../thesis.tex
%*******************************************************************************
%*********************************** First Chapter *****************************
%*******************************************************************************

\chapter{Introduction}  %Title of the First Chapter

\ifpdf
    \graphicspath{{Chapter1/Figs/Raster/}{Chapter1/Figs/PDF/}{Chapter1/Figs/}}
\else
    \graphicspath{{Chapter1/Figs/Vector/}{Chapter1/Figs/}{Chapter1/Figures/}}
\fi


%********************************** %First Section  **************************************
\textit{Probabilistic Inference for Learning  Control} or PILCO \citep{deisenroth2011pilco} is a model-based indirect policy search method for continuous state and action dynamical systems. PILCO learns a probabilistic Bayesian representation of the dynamical system it seeks to govern. Bayesian models explicitly quantify their uncertainty and PILCO uses this information in a principled way to reduce model bias by explicitly incorporating model uncertainty into long-term planning. PILCO also reports unprecedented data-efficiency for a variety of control tasks, such as the cart-pole and cart-double-pole problems. Surprisingly, it does so without any intentional exploration.

PILCO evaluates a particular \textit{policy} by cascading uncertain inputs through its probabilistic representation of the transition dynamics. In doing so, it gains insight into the variety of states that could be visited under a particular policy and is able to make inferences about how effective that policy is at achieving low cost. However, this approach only allows PILCO to make decisions based on the total uncertainty as quantified by the models predictive distribution, when in fact there are multiple sources of uncertainty present in the system: \textit{aleatoric} and \textit{epistemic}. This means that in certain situations PILCO could be making decisions based on uncertainty of which the primary constituent is aleatory which could prohibit learning.

This work investigates how PILCO uses uncertainty in its decision making process by disentangling and quantifying the different sources of uncertainty present in the model of the transition function. Since PILCO is a direct policy search method, the \textit{cost function} is consulted directly, therefore, the influence of the uncertainty in the transition function on the uncertainty in the cost is examined. This is done by using variance as a metric for uncertainty and employing the law of total uncertainty to decompose the total model uncertainty into its constituents. The uncertainties are then estimated by establishing a "gold-standard" Monte-Carlo scheme that propagates trajectories through an approximation to PILCO's dynamics model. The intention of the research is to lay the foundations for an active-exploration scheme to compliment PILCO. This work makes the following contributions:
\begin{itemize}
    \item Presents a variance decomposition that disentangles the sources of uncertainty in the model's predictive distribution that influence PILCO's cost under a particular policy $\pi$.
    \item Creates a “gold standard” Monte Carlo scheme that separates the two sources of uncertainty and quantifies them.
    \item Shows that for complex environments PILCO's policy gradient method can get temporarily stuck in areas of low cost and low epistemic uncertainty and that this slows down learning.
    \item Shows that when PILCO is learning efficiently, it is selecting policies that correspond to low cost and high epistemic uncertainty. 
\end{itemize}

\section{Background} %Section - 1.1 
\label{S:background}
\textit{Reinforcement learning} is a general sequential decision making framework that is best described as learning a mapping from situations to actions in an attempt to maximise a numerical reward signal. The agent, or learner, is not told what to do and so must embark on a mission of trial-and-error to discover actions that produce the most reward. In many cases, the action taken not only influences the instantaneous reward but also the next situation or state, and therefore, all future rewards as a consequence. To find a set of optimal actions given a sequence of situations, the agent must then take into account the effect of an action on all future rewards. These two ideas; trial-and-error search and delayed reward, are the two most distinguishing features of reinforcement learning \citep{sutton2018reinforcement}.

Another highly influential idea is the trade-off between \textit{exploration} and \textit{exploitation}. For an agent to maximise the long-term reward it must favour actions which it has previously tried and found to be successful in yielding high reward. However, in order to discover those actions, it must first have tried actions that it had not previously selected. The agent must \textit{exploit} its current knowledge of the system to gain a high reward but also \textit{explore} new strategies to improve its \textit{policy}. A dilemma arises in that exclusively executing either approach will lead to a failed task. The agent must interchangeably try both approaches and progressively tend towards actions that prove to be better at attaining high reward.

There are two main approaches to reinforcement learning; \textit{model-based} and \textit{model-free} methods. Model-free methods are explicit trial-and-error learners and directly use the experience they gain through interactions with an environment to make decisions. In contrast, model-based techniques use experience indirectly by building a model of the state transition dynamics and reward structure of the environment, and evaluate actions by searching this model \citep{glascher2010states}. Model-free methods are therefore said to rely on \textit{learning} whilst model-based methods primarily  rely on \textit{planning} \citep{sutton2018reinforcement}. 

Recently, model-free methods have achieved impressive performance in a range of complex tasks such as playing Atari games, receiving only pixels and game scores as inputs \citep{mnih2015human}. These methods, however, typically require millions of interactions with the environment before they achieve reasonable performance levels. The large number of required interactions can prohibit the use of these algorithms in some domains; such as mechanical systems with components that quickly wear out \citep{deisenroth2011pilco} or safety-critical systems where carrying out many field trials is prohibitively expensive. With computational power increasing exponentially, the primary bottleneck in the deployment of real-world reinforcement learning applications is fast becoming the number of interactions with the environment \citep{Wan2018ModelbasedRL}.

Model-based methods use the available samples to create a belief about the underlying environment. There are several advantages to this approach, amongst others; the learning process can be more sample-efficient \citep{deisenroth2013gaussian}, prior knowledge and experience can be integrated more easily (particularly when Bayesian models are used) \citep{lopes2012exploration}, and more recently the incorporation of \textit{counterfactual} reasoning (making inferences about actions that were not actually taken) which can be complex to implement without an explicit model-based representation of the environment \citep{buesing2018woulda}. Model-based methods are, however, not without their challenges, building accurate representations of complex real-world environments is difficult and failing to do so can lead to highly suboptimal algorithm behaviour.

Until recently, model-based techniques had not been widely applied to real-world systems. One of the main reasons for this is they can suffer from model bias. This happens when, given a data set of observed state transitions, the learned model incorrectly assumes that it fully describes the environment, when in fact there are many plausible functions that could have generated the data \citep{atkeson1997comparison}\citep{schneider1997exploiting}. This is demonstrated in Fig \ref{Fig:model-bias} where a small data set of transitions are observed (left) and several transition functions exist that could have generated the data (centre). Selecting any single transition function can have severe consequences because predictions are then arbitrary at positions away from the data, but are claimed with full confidence \citep{deisenroth2011pilco}. 
PILCO \citep{deisenroth2011pilco} is a model-based indirect policy search method for continuous state and action dynamical systems. PILCO learns a probabilistic Bayesian representation of the dynamical systems it seeks to govern. Bayesian models explicitly quantify their uncertainty and PILCO uses this information in a principled way to reduce model bias by explicitly incorporating model uncertainty into long-term planning. Fig \ref{Fig:model-bias} (right) shows how PILCO represents the observed data by placing a posterior distribution over the transition function. PILCO also reports unprecedented data-efficiency for a variety of control tasks, such as the cart-pole and cart-double-pole problems. Surprisingly, it does so without any intentional exploration i.e. it is \textit{greedy}. Any exploration that does occur comes from one of three sources: first, the presence of stochasticity in the system; second, the use of a saturating cost function results in the controller favouring uncertain states \citep{deisenroth2013gaussian}; third, future controllers are informed of the performance of past controllers. 

\begin{figure}
\centering    
\includegraphics[width=1.0\textwidth]{Chapter1/Figures/PILCO-model-bias.png}
\caption[Model-based bias in reinforcement learning]{Small data set of observed state transitions (left). Several plausible functions that could have generated the data (centre). Posterior distribution showing model uncertainty (right). Reproduced from \citep{deisenroth2011pilco}.}
\label{Fig:model-bias}
\end{figure}

Attempting to further increase PILCO's data-efficiency would require either more informative prior knowledge of the task or extracting more relevant information from the available data. The addition of a exploration scheme would constitute the latter. Many approaches to learning control realise exploration through introducing randomness into action selection. Some approaches comprise a \textit{random exploration phase}, where the controller generates actions randomly, followed by an \textit{exploitation phase}  \citep{thrun1992active}. However, with this strategy once the exploration phase has finished, the agent is unable to adapt to environmental changes during the purely exploitative phase. Others use $\epsilon$-\textit{greedy} policies, which exploit the action with the maximum expected reward most of the time, but with some probability $\epsilon$ a random action is selected \citep{sutton2018reinforcement}. While $\epsilon$-\textit{greedy} approaches encourage learning, for real-world systems, selecting random actions can repeatedly steer the system towards undesirable or dangerous states. In addition, random action selection can be inefficient and often causes the agent to repeatedly return to already well-explored regions of the state-space because exploration is \textit{undirected}.

\textit{Information-directed} or \textit{active} exploration schemes aim to overcome the inefficiencies and risks associated with undirected exploration by driving exploration towards promising states \citep{zhaohan2019directed}. Since PILCO is primarily designed for learning control of mechanical dynamical systems, it is natural to consider this class of exploration algorithm when attempting to further improve its efficiency. Furthermore, since PILCO already quantifies model uncertainty and uses this uncertainty in long-term planning, it is also logical to attempt to incorporate this information in any considered exploration strategies. 

Currently, PILCO evaluates a particular policy by cascading uncertain inputs through the probabilistic model of the transition dynamics. In doing so, it gains insight into the variety of states that could be visited under that policy and is able to make an informed decision about how effective the policy is at achieving high reward. However, this approach only allows PILCO to make decisions based on the total uncertainty as quantified by the model, when in fact there are multiple sources of uncertainty present in the system. One source is uncertainty due to \textit{aleatory} which is representative of unknowns that differ each time the agent encounters the environment such as measurement noise or chaotic motion in dynamical systems. The other source is \textit{epistemic} uncertainty which is due to things that the agent could know in principle but currently does not. In model-based methods, this can be thought of as lack of knowledge about the system's transition dynamics. Hence, epistemic uncertainty can be reduced by observing more data while aleatoric uncertainty is irreducible. This means that in certain situations PILCO could be making decisions based on uncertainty of which the primary constituent is aleatory. In this case, PILCO could be repeatedly selecting policies corresponding to trajectories associated with high aleatoric uncertainty which could be prohibitive to learning.

This work investigates how PILCO uses uncertainty in its decision making process by disentangling and quantifying the different sources of uncertainty present in the model of the transition function: aleatoric and epistemic uncertainty. Since PILCO is a direct policy search method, the \textit{cost function} is consulted directly, therefore, the influence of the uncertainty in the transition function on the uncertainty in the cost is examined. This is done by using variance as a metric for uncertainty and employing the law of total uncertainty to decompose the total model uncertainty into its constituents. The uncertainties are then estimated establishing a "gold-standard" Monte-Carlo scheme that propagates trajectories through PILCO's dynamics model. The intention of the research is to lay the foundations for a active-exploration scheme to compliment PILCO. 



\nomenclature[z-cif]{$CIF$}{Cauchy's Integral Formula}                                % first letter Z is for Acronyms 
\nomenclature[a-F]{$F$}{complex function}                                                   % first letter A is for Roman symbols
\nomenclature[g-p]{$\pi$}{ $\simeq 3.14\ldots$}                                             % first letter G is for Greek Symbols
\nomenclature[g-i]{$\iota$}{unit imaginary number $\sqrt{-1}$}                      % first letter G is for Greek Symbols
\nomenclature[g-g]{$\gamma$}{a simply closed curve on a complex plane}  % first letter G is for Greek Symbols
\nomenclature[x-i]{$\oint_\gamma$}{integration around a curve $\gamma$} % first letter X is for Other Symbols
\nomenclature[r-j]{$j$}{superscript index}                                                       % first letter R is for superscripts
\nomenclature[s-0]{$0$}{subscript index}                                                        % first letter S is for subscripts

%********************************** %Second Section  *************************************
\section{Related Work} %Section - 1.2
\label{S:relted-work}


\nomenclature[z-PPC]{PPC}{Particles per cell}